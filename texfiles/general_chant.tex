
%% GENERAL %%
\newcommand{\enluminure}[3]{\lettrine[lines=#1]{\usefont{U}{Romantik}{xl}{n} #2}{#3}}

\setlength\parindent{0pt}

\newcommand{\breakpage}{\newpage}

\newcommand{\newSection}[1]{
  \thispagestyle{empty}
  \vspace*{200pt}\par
  {\Huge \centering \bfseries #1\par\nobreak
  \vskip 100pt}
  \breakpage
}

\newcommand{\dualcol}[1]{
    \begin{multicols}{2}
        #1
    \end{multicols}
}


\newcommand{\blackline}[0]{
    \vspace{1cm}
    \hrule
    \vspace{1cm}
}

\newcommand{\bissimple}[1][]{\} bis}

\newcommand{\bisdouble}[2]{
    % The spaces are NECESSARY DON'T ERASE THEM

    \ensuremaths{
        \left.
        \hspace{-0.25cm}
        \begin{tabular}{l}
        #1
        \\#2
        \end{tabular}\right\rbrace
    } \textrm{bis}
    
    % The spaces are NECESSARY DON'T ERASE THEM
}

\newcommand{\bisdoublespace}[2]{
    % The spaces are NECESSARY DON'T ERASE THEM

    \ensuremaths{
        \left.
        \hspace{-0.25cm}
        \begin{tabular}{l}
        #1
        \\#2
        \end{tabular}\right\rbrace
    } \textrm{bis}\\
    \vspace{0.2cm}
    
    % The spaces are NECESSARY DON'T ERASE THEM
}

\newcommand{\bistriple}[3]{
    % The spaces are NECESSARY DON'T ERASE THEM

    \ensuremaths{
        \left.
        \hspace{-0.25cm}
        \begin{tabular}{l}
        #1
        \\#2
        \\#3
        \end{tabular}\right\rbrace
    } \textrm{bis}
    
    % The spaces are NECESSARY DON'T ERASE THEM
}

\newcommand{\bistriplespace}[3]{
    % The spaces are NECESSARY DON'T ERASE THEM

    \ensuremaths{
        \left.
        \hspace{-0.25cm}
        \begin{tabular}{l}
        #1
        \\#2
        \\#3
        \end{tabular}\right\rbrace
    } \textrm{bis}\\
    \vspace{0.2cm}
    
    % The spaces are NECESSARY DON'T ERASE THEM
}

\newcommand{\bisquadruple}[4]{
    % The spaces are NECESSARY DON'T ERASE THEM

    \ensuremaths{
        \left.
        \hspace{-0.25cm}
        \begin{tabular}{l}
        #1
        \\#2
        \\#3
        \\#4
        \end{tabular}\right\rbrace
    } \textrm{bis}
    
    % The spaces are NECESSARY DON'T ERASE THEM
}

\newcommand{\bisquadruplespace}[4]{
    % The spaces are NECESSARY DON'T ERASE THEM

    \ensuremaths{
        \left.
        \hspace{-0.25cm}
        \begin{tabular}{l}
        #1
        \\#2
        \\#3
        \\#4
        \end{tabular}\right\rbrace
    } \textrm{bis}\\
    \vspace{0.2cm}
    
    % The spaces are NECESSARY DON'T ERASE THEM
}

\newcommand{\bisquintuple}[5]{
    % The spaces are NECESSARY DON'T ERASE THEM

    \ensuremaths{
        \left.
        \hspace{-0.25cm}
        \begin{tabular}{l}
        #1
        \\#2
        \\#3
        \\#4
        \\#5
        \end{tabular}\right\rbrace
    } \textrm{bis}
    
    % The spaces are NECESSARY DON'T ERASE THEM
}

\newcommand{\bisquintuplespace}[5]{
    % The spaces are NECESSARY DON'T ERASE THEM

    \ensuremaths{
        \left.
        \hspace{-0.25cm}
        \begin{tabular}{l}
        #1
        \\#2
        \\#3
        \\#4
        \\#5
        \end{tabular}\right\rbrace
    } \textrm{bis}\\
    \vspace{0.2cm}
    
    % The spaces are NECESSARY DON'T ERASE THEM
}

%% HEADER MANAGEMENT %%
%% Using an optional argument other than c in \header allow a horizontal spanning of categories instead of a square one

\newcommand{\header}[2][c]{
    \ifthenelse{\equal{#1}{c}}{
        \begin{minipage}[t]{.77\textwidth}
            \kern0pt
            #2
        \end{minipage}
        \hfill
        \hspace{0.5cm}
        \begin{minipage}[t]{.2\textwidth}
            \kern0pt
            \insertCategories{#1}
        \end{minipage}
    }{
        \ifthenelse{\equal{#1}{v}}{
            \begin{minipage}[t]{.85\textwidth}
                \kern0pt
                #2
            \end{minipage}
            \hfill
            \hspace{0.5cm}
            \begin{minipage}[t]{.15\textwidth}
                \kern0pt
                \insertCategories{#1}
            \end{minipage}
        }{
            \begin{minipage}[t]{.6\textwidth}
                \kern0pt
                #2
            \end{minipage}
            \hfill
            \hspace{0.5cm}
            \begin{minipage}[t]{.4\textwidth}
                \kern0pt
                \insertCategories{#1}
            \end{minipage}
        }
    }
    %\vspace{0.2cm}
}

\renewcommand{\LettrineTextFont}{}

%% COMMENTARY MANAGEMENT

\newcommand{\insertComment}[2]{
    \ifthenelse{\equal{#1}{}}{
        \ifthenelse{\equal{#2}{}}{
        }{
            \subsection{#2}
        }
    }{
        \ifthenelse{\equal{#2}{}}{
            \subsection{#1}
        }{
            \subsection{#1\\#2}
        }
    }
}

%% PDF MANAGEMENT

\hypersetup{
    colorlinks,
    linkcolor={},
    citecolor={},
    urlcolor={}
}

%% SECTION TITLE MANAGEMENT %%

% renew \section to link to the toc

\let\oldcontentsline\contentsline%
\renewcommand\contentsline[4]{%
\hypertarget{toc#4}{}%
\oldcontentsline{#1}{#2}{#3}{#4}}
\let\oldsection\section
\renewcommand\section[1]{%
    \oldsection[#1]{\protect\hyperlink{toc}{#1}}}
    
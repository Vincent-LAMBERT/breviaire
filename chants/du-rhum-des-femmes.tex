\header{
    \section{Du rhum des femmes} \label{du-rhum-des-femmes}
    %
    
    \insertComment{Chanson de Soldat Louis (Renaud Detressan, 1988).}{}
}

\textbf{Refrain :}
\enluminure{4}{\href{https://www.youtube.com/watch?v=XcYng3TGq9E}{D}}{u rhum}, des femmes et d'la bière nom de Dieu
\\Un accordéon pour valser tant qu'on veut
\\Du rhum des femmes, c'est ça qui rend heureux
\\Que l'diable nous emporte
\\On n'a rien trouvé de mieux.
\\Oh oh oh oh, on n'a rien trouvé de mieux.
\\\\Hello ! Cap'taine fait briller tes galons
\\Et reste bien au chaud quand on gèle sur le pont,
\\Nous c'est notre peine qui nous coule sur le front
\\Alors tiens bien les rênes tu connais la chanson.
\\\\Ça fait une paye qu'on n'a pas touché terre
\\Et même une paye qu'on s'fait des gonzesses en poster.
\\Tant pis pour celle qui s'pointera la première
\\J'lui démonte la passerelle, la cale, la lunette arrière.
\\\\Tout est gravé quelque part sur ma peau
\\Tellement que j'en ai les bras comme des romans photos
\\Blessures de guerre, culs d'bouteille, coups de couteaux
\\Tant qu'y aura des comptoirs on aura des héros.
%on parle ici de milles marins.  (distances, l'accord est donc d'occurence.
%s'il sagissait d'un nombre on dirait bien trois mille. mais le cap se compte sur 360 degrés (pas 3000)
\\\\Trois milles du cap et des fois c'est les glandes
\\Quand t'as le coeur qui dérape, t'as les tripes qui fermentent
\\J'essaie de penser aux claques aux filles qui s'impatientent
\\Pas au bateau qui craque entre deux déferlantes
\\

\breakpage
\header{
    \section{Aussitôt que la lumière} \label{aussitot-que-la-lumiere}
    %lien de la publication : https://gallica.bnf.fr/ark:/12148/bpt6k1174609c.image
    \insertComment{Chanson de Maître Adam Billaut (1602-1662).}{}
}

\enluminure{2}{\href{https://www.chansonsaboire.com/index.php?param1=BR197.php}{A}}{ussitôt} que la lumière,
\\A redoré nos coteaux
\\Je commence ma carrière
\\Par visiter mes tonneaux
\\Ravi de revoir l'aurore
\\Le verre en main, je lui dis
\\Vois-tu sur la rive Maure
\\Plus qu'à mon nez de rubis ?
\dualcol{
\\\\Le plus grand roi de la terre
\\Quand je suis dans un repas
\\S'il me déclarait la guerre
\\Ne m'épouvanterait pas
\\A table, rien ne m'étonne
\\Et je pense quand je bois
\\Qui là-haut Jupiter tonne
\\Que c'est qu'il a peur de moi
\\\\Si quelque jour étant ivre
\\La mort arrêtait mes pas
\\Je ne voudrais pas revivre
\\Pour changer ce grand trépas
\\Je m'en irai dans l'Averne
\\Faire enivrer Alecton
\\Et planter une taverne
\\Dans la chambre de Pluton
\\\\De ce nectar délectable
\\Les démons étant vaincus
\\Je ferai chanter au diable
\\Les louanges de Bacchus
\\J'apaiserai de Tantale
\\La grande altération
\\En passant l'onde infernale
\\Je ferai boire Ixion
\\\\Au bout de ma quarantaine
\\Cent ivrognes m'ont promis
\\De venir la tasse pleine
\\Au gîte où l'on m'aura mis
\\Pour me faire une hécatombe
\\Qui signale mon destin
\\Ils arroseront ma tombe
\\De plus de cent brocs de vin
\\\\De marbre et de porphyre
\\Qu'on ne fasse mon tombeau
\\Pour cercueil je ne désire
\\Que le contour d'un tonneau
\\Et veux qu'on peigne ma trogne
\\Avec ces vers à l'entour
\\Ci-gît le plus grand ivrogne
\\Qui jamais ait vu le jour 
}
\breakpage
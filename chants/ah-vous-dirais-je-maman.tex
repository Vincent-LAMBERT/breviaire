\header{
    \headtitle{Ah vous dirais-je maman} \label{ah-vous-dirais-je-maman}
    %
    
    \insertComment{}{}
    %\insertComment{Musique de Mr.Bouin (1761) dans les "Amusements d'une heure et demy". Les premières paroles dattent elles de 1965.}{}
    
    % Le commentaire ne passe pas avec la mise en page
}

\enluminure{4}{\href{https://www.youtube.com/watch?v=KVhCuCiX1Pc}{A}}{h} vous dirai-je maman,
\\A quoi nous passons le temps,
\\Avec mon cousin Eugène,
\\Sachez que ce phénomène,
\\Nous a inventé un jeu
\\Auquel nous jouons tous deux.
\dualcol{
Il m'emmène dans le bois,
\\Et me dit déshabille-toi!
\\Quand je suis nue toute entière,
\\Il me fait coucher par terre
\\Et de peur que je n'ai froid
\\Il vient se coucher sur moi.
\\\\Puis il me dit d'un ton doux:
\\"Ecarte bien tes genoux!"
\\Et - la chose va vous faire rire -
\\Il embrasse ma tirelire.
\\Ah, vous conviendrez maman,
\\Qu'il a des idées vraiment!
\\\\Puis il sort je ne sais d'où,
\\Un p'tit animal très doux:
\\Une sorte de rat sans pattes,
\\Qu'il me donne et que je flatte.
\\Oh le joli petit rat,
\\D'ailleurs il vous le montrera!
\\\\Et c'est juste à ce moment,
\\Que le jeu commence vraiment:
\\Eugène prend sa petite bête,
\\Et la fourre dans une cachette,
\\Qu'il a trouvé le farceur.
\\Où vous situez mon honneur.
\\\\Mais ce petit rat curieux,
\\Très souvent devient furieux:
\\Voilà qu'il sort et qu'il rentre,
\\Et qu'il me court dans le ventre.
\\Mon cousin a bien du mal,
\\A calmer son animal.
\\\\Complètement essoufflé,
\\Il essaie de le rattraper:
\\Moi je ris à perdre haleine,
\\Devant les efforts d'Eugène.
\\Si vous étiez là, maman,
\\Vous ririez pareillement.
\\\\Au bout de quelques instants,
\\Le p'tit rat sort en pleurant:
\\Alors Eugène qui tremblote,
\\Le remet dans sa redingote.
\\Et puis tous deux nous rentrons,
\\Sagement à la maison.
\\\\Mon cousin est merveilleux,
\\Il connaît des tas de jeux:
\\Demain soir sur la carpette,
\\Il doit m'apprendre la levrette.
\\Si vraiment c'est amusant,
\\J'vous l'apprendrai en rentrant.
\\\\Voici ma chère maman,
\\Comment je passe mon temps.
\\Vous voyez je suis très sage,
\\Je fuis tous les bavardages,
\\Et j'écoute vos leçons:
\\Je ne parle pas aux garçons!
}

\breakpage
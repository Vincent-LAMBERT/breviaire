\header{
    \section{Le bateau des vits} \label{le-bateau-des-vits}
    %
    
    \insertComment{Version des frères Jacques (10/06/1965), publiée en 1911.}{}
}
\vspace{-0.3cm}
\enluminure{4}{\href{https://www.youtube.com/watch?v=Djg8PeJzNk0}{U}}{n bateau} chargé de vits
\\Descendait une rivière.
\\Une dame de Paris
\\Voulut en acheter une paire
\dualcol{
\\\\\textbf{Refrain :}
\\Pan, pan ! de la Bretonnière
\\Pan, pan ! de la barbe au con
\\\\Une dame de Paris
\\Voulut s'en acheter une paire
\\Pour en choisir deux jolis
\\Envoya sa chambrière
\\\\Pour en choisir deux jolis
\\Envoya sa chambrière
\\Chambrière, en femme d'esprit
\\S'en est servi la première
\\\\Chambrière, en femme d'esprit
\\S'en est servi la première
\\Elle s'en est si bien servi
\\Qu'elle s'est pété la charnière
\\\\Elle s'en est si bien servi
\\Qu'elle s'est pété la charnière
\\Et du cul jusqu'au nombril
\\Ce n'est plus qu'une vaste ornière
\\\\Et du cul jusqu'au nombril
\\Ce n'est plus qu'une vaste ornière
\\Les morpions nagent dedans
\\Comme poissons en rivière
\\\\Les morpions nagent dedans
\\Comme poissons en rivière
\\On croit baiser par devant
\\Va te faire foutre, c'est par derrière!
\\\\On croit baiser par devant
\\Va te faire foutre, c'est par derrière!
\\On croit lui faire un enfant
\\On ne lui donne qu'un clystère
\\\\On croit lui faire un enfant
\\On ne lui donne qu'un clystère
\\On croit être son amant,
\\On n'est que son apothicaire
\\\\On croit être son amant,
\\On n'est que son apothicaire
\\On croit l'aimer tendrement
\\La marchandise tombe par terre.
\\\\On croit l'aimer tendrement
\\La marchandise tombe par terre.
\\"Ah !" Dit-elle, en l’écrasant,
\\"C’lui là n’battra pas son père."
\\\\"Ah !" Dit-elle, en l’écrasant,
\\"C’lui là n’battra pas son père."
\\"Et tu n’écorcheras pas,
\\Le joli con de ta mère."
}

\breakpage
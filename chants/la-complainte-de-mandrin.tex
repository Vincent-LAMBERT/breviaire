\header[c]{
    \headtitle{La complainte de Mandrin} \label{la-complainte-de-mandrin}
    %
    
    \insertComment{}{}
    %\insertComment{Chant tiré de l'opéra Hyppolite et Aricie (1733).}{}
    % commenté pour raisons de mise en page
}

\enluminure{4}{\href{https://www.youtube.com/watch?v=JCwsASjtryw}{N}}{ous} étions vingt ou trente
\\Brigands dans une bande,
\\Tous habillés de blanc
\\A la mode des, vous m'entendez,
\\Tous habillés de blanc
\\A la mode des marchands.
\dualcol{
La première volerie
\\Que je fis dans ma vie,
\\C'est d'avoir goupillé
\\La bourse d'un,
\\Vous m'entendez,
\\C'est d'avoir goupillé
\\La bourse d'un curé.
\\\\J'entrais dedans sa chambre,
\\Mon Dieu, qu'elle était grande,
\\J'y trouvais mille écus,
\\Je mis la main,
\\Vous m'entendez,
\\J'y trouvais mille écus,
\\Je mis la main dessus.
\\\\J'entrais dedans une autre
\\Mon Dieu, qu'elle était haute,
\\De robes et de manteaux
\\J'en chargeais trois,
\\Vous m'entendez,
\\De robes et de manteaux
\\J'en chargeais trois chariots.
\\\\Je les portais pour vendre
\\A la foire de Hollande
\\J'les vendis bon marché
\\Ils m'avaient rien,
\\Vous m'entendez,
\\J'les vendis bon marché
\\Ils m'avaient rien coûté.
\\Ces messieurs de Grenoble
\\Avec leurs longues robes
\\Et leurs bonnets carrés
\\M'eurent bientôt,
\\Vous m'entendez,
\\Et leurs bonnets carrés
\\M'eurent bientôt jugé.
\\\\Ils m'ont jugé à pendre,
\\Que c'est dur à entendre
\\A pendre et étrangler
\\Sur la place du,
\\Vous m'entendez,
\\A pendre et étrangler
\\Sur la place du marché.
\\\\Monté sur la potence
\\Je regardais la France
\\Je vis mes compagnons
\\A l'ombre d'un,
\\Vous m'entendez,
\\Je vis mes compagnons
\\A l'ombre d'un buisson.
\\\\Compagnons de misère
\\Allez dire à ma mère
\\Qu'elle ne m'reverra plus
\\J'suis un enfant,
\\Vous m'entendez,
\\Qu'elle ne m'reverra plus
\\J'suis un enfant perdu.}

\breakpage
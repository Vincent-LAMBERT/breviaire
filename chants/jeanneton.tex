\header{
    \section{Jeanneton} \label{jeanneton}
    %
    
    \insertComment{Variante de "Brunettes et Petits Airs tendres", publiée en 1703 par Christophe Ballard.}{}
}
% 
\enluminure{4}{\href{https://www.youtube.com/watch?v=e_tXAwl-E0Y}{J}}{eanneton} prend sa faucille
\\Larirette, larirette
\\Jeanneton prend sa faucille
\\Pour aller couper le jonc. ~\bissimple
\\\\En chemin elle rencontre
\\Quatre jeunes et beaux garçons.~~~~~~~~ \bissimple
\\\\Le premier, un peu timide,
\\Lui caressa le menton. ~~~~~~~~~~~~~~~~~~~~~\bissimple
\\\\Le second, un peu moins sage,
\\La coucha sur le gazon. ~~~~~~~~~~~~~~~~~~~~\bissimple
\\\\Le troisième, encore moins sage,
\\Lui souleva le jupon. ~~~~~~~~~~~~~~~~~~~~~~~~~\bissimple
\\\\Ce que fit le quatrième
\\N'est pas dit dans la chanson. ~~~~~~~~~~~~~\bissimple
\\\\La morale de cette histoire,
\\C'est qu'les hommes sont des cochons. ~~\bissimple
\\\\La morale de cette morale,
\\C'est qu'les femmes aiment les cochons. ~\bissimple
\\\\La morale de ces morales,
\\C'est qu'sur quatre, y'a trois couillons. ~~~\bissimple
\\\\
\\\\
\\
\breakpage
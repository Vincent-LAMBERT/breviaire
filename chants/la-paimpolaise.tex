\header{
    \section{La paimpolaise} \label{la-paimpolaise}
    %
    \insertComment{Parodie d'une chanson écrite en 1784 à Paris par un dinannais inconnu. Reprise l'an suivant par Félix Mayol}{}
}

\enluminure{4}{\href{https://www.youtube.com/watch?v=j38QMazDQCQ}{R}}{ien} à foutre de l'armée de France
\\Ni de la marine de Toulon
\\Nous voulons notre indépendance
\\Vive le Front Libéral Breton.
\\\\J'aime Paimpol et sa falaise
\\Et le Front de Libération
\\Mais chez nous y'a comme un malaise
\\Faut qu'ça pète au pays breton.
\\\\J'aime Paimpol et sa falaise
\\Son calvaire et son vieux pardon
\\Mais c'que j'aime c'est ma paimpolaise
\\Qui m'attend au pays breton.
\\\\\textbf{Version Nationale :}
\\La Bretagne c'est pas la France
\\C'est vraiment un pays de cons.
\\Donnez-leur leur indépendance
\\Qu'ils nous lâchent enfin les roustons.

\breakpage
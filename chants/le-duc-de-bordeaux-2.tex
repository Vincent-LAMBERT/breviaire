\header{
    \section{Le duc de Bordeaux  (2ème version)} \label{le-duc-de-bordeaux-2}
    %
    
    \insertComment{L'air serait composé par Marc-Antoine, marquis de Dampierre (début 17ème).}{Les paroles ont été écrites de 1622 à 1733.}
}
%
\enluminure{4}{\href{https://www.youtube.com/watch?v=JGLG07JAmOg}{L}}{e duc} de Bordeaux ressemble à son frère
\\Son frère à son père et son père à mon cul
\\De là j'en conclus qu'le duc de Bordeaux
\\Ressemble à mon cul comme deux gouttes d'eau.
\\\\Le duc de Chevreuse ayant déclaré
\\Que tous les cocus devaient être noyés
\\Madame de Chevreuse lui a demandé
\\S'il était bien sûr de savoir nager.
\\\\La duchesse de la Trémouille
\\Malgré sa très grande pitié,
\\A patiné plus de paires de couilles,
\\Que la Grande Armée n'a usé de souliers.
\\\\Le roy Dagobert a un' pine en fer,
\\Le bon Saint-Eloi lui dit: "Eh bien! mon roi,
\\Si vous m'enculez, vous m'écorcherez"
\\"C'est vrai, dit le roy, j'en f'rai faire un' de bois".
\\\\J'emmerde le roy et le comt' d'Artois,
\\Le duc de Berry et la duchesse aussi;
\\Le duc de Nemours, j' l'emmerde à son tour
\\Le duc d'Orléans, je l'emmerde en mêm' temps!
\\\\"Chasseur as-tu vu le trou de mon cul ?
\\Si tu veux le voir, tu reviendras ce soir
\\Moi j'ai vu le tien, je n'en ai rien dit ;
\\Si tu vois le mien, tu n'en diras rien."
\\\\"Nom de Dieu, disait la princesse
\\En voyant la pine du baron,
\\J'aimerais mieux l'avoir dans les fesses
\\Que de la voir dans son pantalon"

\breakpage
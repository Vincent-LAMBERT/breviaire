\header{
    \section{Pelot d'Hennebont} \label{pelot-d-hennebont}

    \insertComment{Reprise par Tri Yann sur la base d'une chanson gallèse de Haute-Bretagne, Pelot de Betton datant de la fin du 18ème siècle. La mélodie a un rythme d'an-dro, danse de Basse-Bretagne.}{}
    % ... 
    
    %
}

\enluminure{3}{\href{https://www.youtube.com/watch?v=ne2LzSG8bic}{M}}{a chère} maman je vous écris
\\Que nous sommes entrés dans Paris
\\$\left.\begin{tabular}{l}
\hspace{-0.4cm}
Que je sommes déjà Caporal
\\
\hspace{-0.4cm}
Et serons bientôt Général ~~~~~
\end{tabular}\right\rbrace$ bis
\\\\A la bataille, je combattions 
\\Les ennemis de la nation 
\bisdoublespace{Et tous ceux qui se présentions}
{A grand coups de sabres les émondions ~~~~~~}
\\Le roi Louis m'a z'appelé
\\C'est "sans quartier" qu'il m'a nommé
\bisdoublespace{Mais "sans quartier", c'est point mon nom,}
{J'lui dit "j'm'appelle Pelot d'Hennebont" ~~~~}
\\J'y aquiris un biaux ruban
\\Et je n'sais quoi au goût d'argent
\bisdoublespace{Il dit boute ça sur ton habit}
{Et combats toujours l'ennemi ~~~~~~~~~~~~~~~~~~~~}
\\Faut qu'ce soye que'que chose de précieux
\\Pour que les autres m'appellent monsieur
\bisdoublespace{Et foutent lou main à lou chapiau}
{Quand ils veulent conter au Pelot ~~~~~~~~~~~~~~}
\\Ma mère si j'meurs en combattant
\\J'vous enverrais ce biau ruban
\bisdoublespace{Et vous l'foutrez à votre fusiau}
{En souvenir du gars Pelot ~~~~~~~~~~~~~~~~~~~~~~~~~}
\\Dites à mon père, à mon cousin
\\A mes amis que je vais bien
\bisdoublespace{Je suis leur humble serviteur}
{Pelot qui vous embrasse le coeur ~~~~~~~~~~~~~~~~}
\breakpage
\header{
    \section{Réveillez-vous Picards} \label{reveillez-vous-picards}
    %
    
    \insertComment{Réveillez-vous Picards est actuellement l'hymne régional picard.}{Il serait issu de l'air chanté par les bandes de Picardie (à l'origine entre autres du régiment de Picardie) avant 1479 et leur rattachement à la couronne de France.}
}

\enluminure{4}{\href{https://www.youtube.com/watch?v=YnXR3CeKepg}{R}}{éveillez-vous} Picards,
\\Picards et Bourguignons.
\\Apprenez la manière d'avoir de bons bâtons,
\\Car voici le printemps et aussi la saison
\\Pour aller à la guerre donner des horions.
\\\\Tel parle de la guerre
\\Mais ne sait pas que c'est:
\\Je vous jure mon âme que c'est un piteux faict
\\Et que maints hommes d'armes et gentils compagnons
\\Y ont perdu la vie, et robe et chaperon.
\\\\Où est ce duc d'Autriche?
\\Il est en Pays-Bas
\\Il est en Basse Flandre avec ses Picards
\\Qui nuit et jour le prient qu'il les veuille mener
\\En la Haute Bourgogne pour la lui contester.
\\\\Quand serons en Bourgogne,
\\Et en Franche Comté,
\\Ce sera qui-qu'en-grogne le temps de festoyer
\\Bout'ront le roy de France, dehors de ces costeaux
\\Et mettrons dans nos panses le vin de leurs tonneaux
\\\\Adieu, adieu, Salins,
\\Salins de Besançon
\\Et la ville de Beaulne, là où les bons vins sont
\\Les Picards les ont bus, les Flamands les paieront
\\Quatre pastars la pinte ou bien battus seront.
\breakpage
Nous lansquenets et reîtres
\\Et soudards si marchons
\\Sans finir de connaître où nous arriverons,
\\Aidons Dame Fortune et destin que suivons
\\A prêter longue vie aux soldats Bourguignons.
\\\\Quand mourrons de malheur
\\La hacquebutte au poing
\\Que Duc notre Seigneur digne tombeau nous doint
\\Et que dedans la terre où tous nous en irons
\\Fasse le repos guerre aux braves bourgignons
\\\\Et quand viendra le temps
\\Où trompes sonneront
\\Au dernier ralliement, quand nos tambours battront
\\Nous lèveront bannières aux fusils bourgignons
\\Pour aller à la guerre donner des horions.
\\
\bigskip
\begin{center}
\includegraphics[width=0.9\textwidth]{images/picard.jpg}
\end{center}

\breakpage
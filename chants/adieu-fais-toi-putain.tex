\header{
    \section{Adieu fais-toi putain} \label{adieu-fais-toi-putain}
    %
    
    \insertComment{Publiée dans le "Panier aux ordures" (1866) sous le nom de "Crème des vertus".}{Parodie de la chanson "La grâce des dieux".}
}

\enluminure{4}{\href{https://www.youtube.com/watch?v=n-9HxjCTKUY}{T}}{u} vas quitter ta bonne mère, 
\\Pour t'en aller dans un boxon.
\\Je ne te retiens pas ma chère,
\\Si c'est là ta vocation.
\\Suis bien les conseils de ta mère, 
\\Avant toi je fis ce métier.
\\Tu n'as jamais connu ton père, 
\\C'était peut-être tout le quartier.
\\\\Adieu, fais-toi putain. 
\\Va t'en gagner ton pain. 
\\Adieu! Ma fille, adieu! 
\\A la grâce de Dieu!
\\\\Evite surtout la vérole,
\\Chancres, poulains, et caetera ... 
\\Et ne crois jamais, sur ma parole, 
\\Le fouteur qui te baisera. 
\\Regarde bien si sa culotte
\\Cache un vit bien entretenu. 
\\Découvre toujours la calotte, 
\\Avant de lui prêter ton cul.
\\\\Respecte bien la maquerelle, 
\\N'offense pas le maquereau.
\\Tâche de te conserver belle,
\\Et surtout n'épargne pas l'eau. 
\\Trois fois par jour, dans la cuvette, 
\\Lave ton cul bien proprement,
\\Et dans la table de toilette, 
\\Que l'onguent soit abondant.

\breakpage
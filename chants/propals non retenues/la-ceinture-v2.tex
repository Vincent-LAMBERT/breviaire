\header{
    \section{La ceinture} \label{la-ceinture-v2}
    %
    \insertComment{Parodie de "le départ pour la Syrie" de M. De Laborde.}{}
}

\enluminure{2}{\href{https://www.youtube.com/watch?v=YX6SOj86ABk}{P}}{artant} pour la croisade, un Sire fort jaloux
\\De l'honneur de son nom et de son droit d'époux,
\\Fit faire une ceinture à solide fermoir
\\Qu'il attacha lui-même à sa femme un beau soir.
\\\\\textbf{Refrain :}
\\Tralalalalalère, Tralalalalala,
\\Tralalalalalère, Tralalalalala.
\\\\Une fois son honneur solidement bouclé,
\\Le Sire s'en alla en emportant la clef
\\Depuis la tendre Yseult soupire nuit et jour:
\\"Quand donc t'ouvriras-tu, prison de mes amours?"
\\\\Elle fit la rencontr' le soir au fond d'un bois,
\\D'un jeune troubadour, poète montmartrois,
\\Elle lui demanda gentiment d'essayer
\\Si d'un poèt' l'amour peut faire un serrurier.
\\\\Elle était désirable et belle tant et tant,
\\Que le fermoir céda et qu'elle en fit autant
\\Depuis bientôt deux ans durait leur tendre amour,
\\Quand le seigneur revint avec corn's et tambours.
\\\\La belle étant enceint' depuis bientôt neuf mois,
\\S'écria: "Sur ma vie, quel malheur j'entrevois,
\\En mettant la ceinture et la serrant un peu
\\Notre seigneur jaloux n'y verra que du feu".
\\\\Le sir' s'en aperçut et se mit en courroux,
\\"Seigneur, s'écria-t-ell', cet enfant est de vous!
\\Depuis votre départ, votre fils enfermé
\\Attend votre retour, pour être délivré".
\\\\"Miracle, cria-t-il, femme au con vertueux,
\\Ouvrons vite la porte au fils respectueux!"
\\De joie, la tendre Yseult, à ces mots, enfantait
\\Et depuis, la ceintur', c'est lui qui s' la mettait.


\breakpage
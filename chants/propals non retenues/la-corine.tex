\header{
    \headtitle{La Corine} \label{la corine}
    %
    \insertComment{}{}
}

\enluminure{2}{\href{https://www.youtube.com/watch?v=oCOag98Vvl4}{Q}}{uand} la chouette aux yeux jaunes la nuit en plein mois d'août
\\D'un long cri qui résonne appelle son hibou
\\Les braves gens du village la mine réjouie
\\Savent bien que ce ramage, c'est pas l'oiseau de nuit
\\\\Ils disent "Tiens c'est la Corinne
\\Qu'a encore trouvé une pine
\\La petite noire du garde chasse
\\C'est un vrai piège à bécasses"
\\A l'unisson les paroissiens
\\Dirent "Y a qu'ça qui lui fait du bien"
\\\\Et quand l'hiver s'en vient sous les premiers flocons
\\D'un grand coup de surin on saigne le cochon
\\Est-ce la bête qui agonise de qui proviennent ces cris
\\Poussés derrière l'église ? Mais les fidèles qui prient
\\\\Disent "Tiens c'est la Corinne
\\Qu'a encore trouvé une pine
\\C'est celle du berger Bobby
\\Ça lui change de ses brebis"
\\A l'unisson les paroissiens
\\Dirent "Y a qu'ça qui lui fait du bien"
\\\\Elle amena en Afrique son mari Casimir
\\Qui cru entendre un soir un éléphant barrir
\\Mais ce long cri sauvage cette féroce clameur
\\Les guerriers du village la connaissaient par cœur
\\\\Ils dirent "Tiens c'est la Corinne
\\Qu'a encore trouvé une pine
\\Sûrement celle du grand sorcier
\\Qui lui agite le couscoussier"
\\A l'unisson les Africains
\\Dirent "Y a qu'ça qui lui fait du bien"
\\\\Dans un transatlantique sur le chemin du retour
\\Ils croisèrent des baleines poussant des cris d'amour
\\Mais dans la nuit obscure ces cris de supplicié
\\Les marins les reconnurent ainsi que le plaisanciers
\\\\Ils dirent "Tiens c'est la Corinne
\\Le capitaine la taquine
\\A cette heure là en principe
\\Il lui fait fumer sa pipe"
\\A l'unisson tous les marins
\\Dirent " Y a qu'ça qui lui fait du bien"
\\\\Sentant la mort prochaine elle dit à son époux
\\J'veux un cercueil de chêne avec des nœuds partout
\\Cette innocente prière fut bien sûr exaucée
\\Depuis lors au cimetière quand on entend glousser
\\\\On dit "Tiens c'est la Corinne
\\Qu'a encore trouvé une pine
\\Ses amants n'avaient pas tort
\\Elle peut faire bander un mort"
\\A l'unisson les paroissiens
\\Dirent "Y a qu'ça qui lui fait du bien"
\\\\Cette vie dissolue l'amena pourtant au ciel
\\Pour affronter les foudres du bon Père éternel
\\Reçue par le concierge elle poussa un long cri
\\En empoignant sa barbe mais les anges ont souris
\\\\Ils dirent "Tiens c'est la Corinne
\\Qu'a encore trouvé une pine
\\C'est Saint Pierre à tous les coups
\\Qui essaye son passe-partout"
\\Et le Bon Dieu a dit "Nom d'un chien
\\Faudra que j'essaye ça un matin"
\\Et le Bon Dieu a dit "Nom d'un chien
\\Ça ne peut que me faire du bien"
\breakpage
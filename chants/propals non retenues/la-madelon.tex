\header{
    \section{La Madelon} \label{la-madelon}
    %
    \insertComment{Chanson de Charles-Joseph Pasquier dit Bach, datant du 19/03/1914.}{Paroles de Louis Bousquet, musique de Camille Robert.}
}

\enluminure{2}{\href{https://www.youtube.com/watch?v=hdDpqLnMLww}{P}}{our} le repos, le plaisir du militaire
\\Il est là-bas à deux pas de la forêt
\\Une maison aux murs tout couverts de lierre
\\Au Tourlourou c'est le nom du cabaret
\\La servante est jeune et gentille
\\Légère comme un papillon
\\Comme son vin son œil pétille
\\Nous l'appelons la Madelon
\\Nous en rêvons la nuit, nous y pensons le jour
\\Ce n'est que Madelon mais pour nous c'est l'amour
\\\\\textbf{Refrain :}
\\Quand Madelon vient nous servir à boire
\\Sous la tonnelle on frôle son jupon
\\Et chacun lui raconte une histoire
\\Une histoire à sa façon
\\La Madelon pour nous n'est pas sévère
\\Quand on lui prend la taille ou le menton
\\Elle rit, c'est tout le mal qu'elle sait faire
\\Madelon, Madelon, Madelon
\\\\Nous avons tous au pays une payse
\\Qui nous attend et que l'on épousera
\\Mais elle est loin, bien trop loin pour qu'on lui dise
\\Ce qu'on fera quand la classe rentrera
\\En comptant les jours on soupire
\\Et quand le temps nous semble long
\\Tout ce qu'on ne peut pas lui dire
\\On va le dire à Madelon
\\On l'embrasse dans les coins. Elle dit : "Veux-tu finir..."
\\On s'figure que c'est l'autre, ça nous fait bien plaisir.
\\\\Un caporal en képi de fantaisie
\\S'en fut trouver Madelon un beau matin
\\Et, fou d'amour, lui dit qu'elle était jolie
\\Et qu'il venait pour lui demander sa main
\\La Madelon, pas bête, en somme
\\Lui répondit en souriant
\\Et pourquoi prendrais-je un seul homme
\\Quand j'aime tout un régiment
\\Tes amis vont venir, tu n'auras pas ma main
\\J'en ai bien trop besoin pour leur verser du vin


\breakpage
\header{
    \headtitle{En passant par la Lorraine} \label{en-passant-par-la-lorraine}
    %
    \insertComment{Chanson bretonne iprimée en 1535 par Roland de Lassus. Georges Brassens y fait allusion dans les sabots d'Hélène.}{Vilaine à le sens de paysane ou de laide (jeu de mot).}
}

\enluminure{2}{\href{https://www.youtube.com/watch?v=7H7VGwT_0YU}{E}}{n passant} par la Lorraine
\\Avec mes sabots,
\\En passant par la Lorraine
\\Avec mes sabots,
\\Rencontrai trois capitaines,
\\Avec mes sabots, dondaine,
\\Oh ! Oh ! Oh ! Avec mes sabots.
\\\\Rencontrai trois capitaines,
\\avec mes sabots,
\\Rencontrai trois capitaines,
\\Avec mes sabots,
\\Ils m'ont appelée vilaine,
\\Avec mes sabots, dondaine,
\\Oh ! Oh ! Oh ! Avec mes sabots.
\\\\Ils m'ont appelée vilaine,
\\Avec mes sabots,
\\Ils m'ont appelée vilaine,
\\Avec mes sabots,
\\Je ne suis pas si vilaine,
\\Avec mes sabots, dondaine,
\\Oh ! Oh ! Oh ! Avec mes sabots.
\\Je ne suis pas si vilaine,
\\Avec mes sabots,
\\Je ne suis pas si vilaine,
\\Avec mes sabots,
\\Le fils aîné du roi m'aime,
\\Avec mes sabots, dondaine,
\\Oh ! Oh ! Oh ! Avec mes sabots.
\\\\Le fils aîné du roi m'aime,
\\Avec mes sabots,
\\Le fils aîné du roi m'aime,
\\Avec mes sabots,
\\M'a donné pour mes étrennes,
\\Avec mes sabots, dondaine,
\\Oh ! Oh ! Oh ! Avec mes sabots.
\\\\M'a donné pour mes étrennes,
\\Avec mes sabots
\\M'a donné pour mes étrennes,
\\Avec mes sabots
\\Un bouquet de marjolaine,
\\Avec mes sabots, dondaine,
\\Oh ! Oh ! Oh ! Avec mes sabots.
\\\\Un bouquet de marjolaine,
\\Avec mes sabots,
\\Un bouquet de marjolaine,
\\Avec mes sabots,
\\S'il fleurit je serai sienne,
\\Avec mes sabots, dondaine,
\\Oh ! Oh ! Oh ! Avec mes sabots.
\\\\S'il fleurit je serai sienne,
\\Avec mes sabots,
\\S'il fleurit je serai sienne,
\\Avec mes sabots,
\\Mais s'il meurt, je perds ma peine,
\\Avec mes sabots, dondaine,
\\Oh ! Oh ! Oh ! Avec mes sabots. 

\breakpage
\header{
    \headtitle{Loguivy de la mer} \label{loguivy-de-la-mer}
    %
    \insertComment{Chanson de François Budet (1965). Décrit le port de pêche de Loguivy, à Ploubazlanec. }{Elle est désormais considérée comme un des grands chants de marins contemporains.}
}

\enluminure{2}{\href{https://www.youtube.com/watch?v=TBfstyjWGA8}{I}}{ls reviennent} encore à l'heure des marées
\\S'asseoir sur le muret le long de la jetée
\\Ils regardent encore au-delà de Bréhat
\\Respirant le parfum du vent qui les appelle
\\Mais s'il est révolu le temps des Terres-Neuvas
\\La race des marins chez nous ne s'en va pas
\\\\\textbf{Refrain :}
\\Loguivy-de-la-Mer Loguivy-de-la-Mer
\\Tu regardes mourir les derniers vrais marins
\\Loguivy-de-la-Mer au fond de ton vieux port
\\S'entassent les carcasses des bateaux déjà morts
\\\\Ils ont connu le temps où la voile était reine
\\Ils parlent de haubans de focs et de misaines
\\De tout ce qui a fait le charme de leur vie
\\Et qu'ils emporteront avec eux dans l'oubli
\\Mais s'il est révolu le temps des Cap-Horniers
\\Il reste encore chez nous d'la graine d'aventurier
\\\\Je n'ai jamais su dire ce que disent leurs yeux
\\Perdus dans ces visages burinés par le vent
\\Ces beaux visages d'hommes ces visages de vieux
\\Qui savent encore sourire et dire à nos vingt ans
\\Remettez vos cabans et rompez les amarres
\\Allez-y de l'avant mais tenez bon la barre.
\breakpage
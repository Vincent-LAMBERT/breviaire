\header{
    \headtitle{Ah le petit vin blanc} \label{ah-le-petit-vin-blanc}
    %
    \insertComment{Paroles de Jean Dréjac, composée par Charles Borel-Clerc (1943) chantée par Michèle Dorlan.}{}
}

\enluminure{2}{\href{https://www.youtube.com/watch?v=IDmPxv3hUaI}{V}}{oici} le printemps
\\La douceur du temps
\\Nous fait des avances
\\Partez mes enfants
\\Vous avez vingt ans
\\Partez en vacances
\\Vous verrez agiles
\\Sur l'onde tranquille
\\Les barques dociles
\\Au bras des amants
\\De fraîches guinguettes
\\Des filles bien faites
\\Y a des chansonnettes
\\Et y a du vin blanc
\\\\\textbf{Refrain :}
\\Ah, le petit vin blanc
\\Qu'on boit sous les tonnelles
\\Quand les filles sont belles
\\Du côté de Nogent
\\Et puis de temps de temps
\\Un air de vieille romance
\\Semble donner la cadence
\\Pour fauter, pour fauter
\\Dans les bois, dans les prés
\\Du côté, du côté de Nogent
\breakpage
Suivant le conseil
\\Monsieur le Soleil
\\Connaît son affaire
\\Cueillons, en chemin
\\Ce minois mutin
\\Cette robe claire
\\Venez belle fille
\\Là, sous la charmille
\\Soyez bien gentille
\\L'amour nous attend
\\Les tables sont prêtes
\\L'aubergiste honnête
\\Y a des chansonnettes
\\Et y a du vin blanc
\\\\À ces jeux charmants
\\La taille souvent
\\Prend de l'avantage
\\Ce n'est pas méchant
\\Ça finit tout le temps
\\Par un mariage
\\Le gros de l'affaire
\\C'est lorsque la mère
\\Demande, sévère
\\À la jeune enfant
\\"Ma fille raconte
\\Comment, triste honte
\\As-tu fait ton compte?
\\Réponds, je t'attends"
\\\\\textbf{Variation}
\\\\Car c'est toujours pareil
\\Tant qu'y aura du soleil
\\On verra les amants au printemps
\\S'en aller pour fauter
\\Dans les bois, dans les prés
\\Du côté, du côté de Nogent

\breakpage
\header{
    \section{Santiano} \label{santiano}
    %
    \insertComment{Chanson de Hugues Aufray, traduite de l'anglais "Santiana".}{Santiano fait référence à sainte Anne, la patronne de la Bretagne et de ses marins.}
}

\enluminure{4}{\href{https://www.youtube.com/watch?v=qYyaKNvcKEA}{C}}{'est} un fameux trois-mâts, fin comme un oiseau
\\(Hissez haut ! Santiano !)
\\Dix-huit nœuds, quatre cents tonneaux
\\Je suis fier d'y être matelot
\\\\\textbf{Refrain :}
\\Tiens bon la vague et tiens bon le vent
\\Hissez haut ! Santiano !
\\Si Dieu veut, toujours droit devant
\\Nous irons jusqu'à San Francisco
\\\\Je pars pour de longs mois en laissant Margot
\\(Hissez haut ! Santiano !)
\\D'y penser, j'avais le cœur gros
\\En doublant les feux de Saint Malo
\\\\On prétend que là-bas, l'argent coule à flots
\\(Hissez haut ! Santiano !)
\\On trouve l'or au fond des ruisseaux
\\J'en ramènerai plusieurs lingots
\\\\Un jour je reviendrai, chargé de cadeaux
\\Hissez haut ! Santiano !
\\Au pays, j'irai voir Margot
\\À son doigt, je passerai l'anneau
\\\\\textbf{Refrain final :}
\\Tiens bon le cap et tiens bon le flot
\\Hissez haut ! Hissez haut ! Santiano !
\\Sur la mer qui fait le gros dos
\\Nous irons jusqu'à San Francisco


\breakpage


\header{
    % \section{La Faluche} \label{la-faluche}
    %
    \headtitle{La Faluche} \label{la-faluche}
    
    \insertComment{Sur l'air de "La rumeur" d'Yves Duteil}{Chanson écrite par Big Mama.}
}
    
\enluminure{4}{\href{https://www.youtube.com/watch?v=TV5wnbRYI08}{L}}{a faluche} est accueillante
\\Et s'exprime à travers nous.
\\C'est la coiffe étudiante
\\Et si belle avant tout.
\dualcol{
    Un jour elle est apparue,
    \\Commençant sa conquête,
    \\Et depuis elle a vécu
    \\En ornant toutes les têtes.
    \\\\La faluche a un cœur,
    \\Elle se nourrit de présents.
    \\Elle n'aura jamais peur,
    \\Elle grandit avec le temps.
    \\\\Mais elle a ses détracteurs
    \\Qui ne font que critiquer,
    \\Sans savoir, accusateurs,
    \\Ils ne font que l'insulter.
    \\\\C'est à nous, après coup, 
    \\De se justifier partout,
    \\D'expliquer les traditions,
    \\La base de not' passion.
    \\\\La faluche est un microbe
    \\Qui s'transmet avec amour,
    \\Qui répond à un code
    \\Que nous respect'rons toujours.
    \\\\Les autres font des grimaces,
    \\Mais la faluche est tenace,
    \\Elle s'infiltre, elle s'étend,
    \\Elle s'engouffre, elle se répand.
    \\\\C'est de l'or, c'est du miel
    \\On la croit tombée du ciel.
    \\Jamais nul ne saura
    \\A quel point on aime ça.
    \\\\C'est bien plus fort que l'amitié,
    \\C'est une grande fraternité,
    \\Et c'est beau, et c'est vrai:
    \\Vive la solidarité.
    \\\\Entre amis, plus on rit,
    \\Plus elle chante et se réjouit;
    \\Et aimer, s'éclater:
    \\C'est encore la propager.
    \\\\On s'amuse sans raison,
    \\Pour un oui, pour un non,
    \\Quelle que soit la façon,
    \\Il suffit que ce soit bon.
    \\\\C'est un hommage aux plaisirs,
    \\À Bacchus et à Rabelais,
    \\À la vie, aux désirs,
    \\C'est la fête, c'est le pied.
    \\\\Elle est toujours parmi nous,
    \\Dans nos gestes, dans nos mots,
    \\On se reconnaît partout,
    \\La faluche est not' flambeau.
    \\\\Tous les ans, on lui souhaite 
    \\Un joyeux anniversaire
    \\Et on chante à tue-tête 
    \\Pour celle qui a su nous plaire.
    \\\\La faluche qui est venue 
    \\Ne partira jamais plus.
    \\Dans nos cœurs, le bonheur 
    \\Ne s'en ira pas non plus.
}

\blackline

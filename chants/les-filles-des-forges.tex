\header{
    \headtitle{Les filles des forges} \label{les-filles-des-forges}
    %
    \insertComment{Chanson du XVIIIème siècle chantée et dansée à la Saint-Eloi, venant de Paimpont (en Bretagne) un commune connue pour ses forges.}{}
}

\enluminure{3}{\href{https://www.youtube.com/watch?v=SpJZRTsDHjM}{D}}{igue, }ding don, don, ce sont les filles des forges \bissimple\\
$\left.\begin{tabular}{l}
\hspace{-0.4cm}
Des forges de Paimpont, digue ding dondaine
\\
\hspace{-0.4cm}
Des forges de Paimpont, dingue ding dondon 
\end{tabular}\right\rbrace$ bis
\\\\\\Digue, ding don, don, elles s'en vont à confesse \bissimple
\bisdoublespace{Au curé du canton, digue ding dondaine}
{Au curé du canton, dingue ding dondon}
\textbf{De même :}
\\... Qu'avez-vous fait les filles
\\Pour demander pardon ...
\\\\J'avions couru les bals
\\Et les jolis garçons, 
\\\\Ma fille pour pénitence
\\Nous nous embrasserons,
\\\\Digue, ding don, don, je n'embrasse point les prêtres \bissimple
\bisdoublespace{Mais les jolis garçons, digue ding dondaine}
{Qu'ont du poil au menton, dingue ding dondon}
... Ce sont les filles des forges
\\Des forges de Paimpont ...

\breakpage
\header{
    \headtitle{Fanchon} \label{fanchon}
    %
    
    \insertComment{Chanson premièrement publiée en 1726 dans le "Recueil des plus belles chansons".}{}
}

\enluminure{4}{\href{https://www.youtube.com/watch?v=nZVdImxRlGA}{A}}{mis} il faut faire une pause
\\J'aperçois l'ombre d'un bouchon
\\Buvons à l'aimable Fanchon
\\Chantons pour elle quelque chose
\\\\\textbf{Refrain :}
\\Ah! Que son entretien est doux
\\Qu'elle a de mérite et de gloire
\bisdouble{Elle aime à rire, elle aime à boire}
{Elle aime à chanter comme nous...}
Oui comme nous! \} ter
\\\\Fanchon, quoique bonne chrétienne
\\Fut baptisée avec du vin
\\Un Bourguignon fut son parrain
\\Une Bretonne sa marraine
\\\\Fanchon préfère la grillade
\\A d'autres mets plus délicats
\\Son teint prend un nouvel éclat
\\Quand on lui verse une rasade
\\\\Fanchon ne se montre cruelle
\\Que lorsqu'on lui parle d'amour
\\Mais moi je ne lui fais la cour
\\Que pour m'enivrer avec elle.
\\\\Un jour, le copain La Grenade
\\Lui mit la main dans son corset
\\Elle répondit par un soufflet
\\Sur le museau du camarade
\breakpage

%paroles originales de 1726
% 1. Amis il nous faut faire pose,
%J'aperçois l'ombre d'un bouchon,
%Bûvons à l'aimable Fanchon,
%Faisons pour elle quelque chose ;
%2. Si quelque fois elle est cruelle,
%C'est quand on luy parle d'amour,
%Pour moy je ne lui fais ma cour,
%Que pour badiner avec elle ;
%3. Elle préfère une grillade
%Aux ragouts les plus délicats,
%Son teint prend un nouvel éclat,
%Lorsqu'elle tient une rasade ;
%4. A table d'une humeur aimable,
%De Bacchus elle est le soûtien,
%Les beaux mots ne luy coûte rien,
%Elle en dit des verts & des meurs ;
%5. Recevons-la, tout nous empresse,
%Elle mérite d'être avec nous,
%Elle boit mieux qu'aucun d'entre nous,
%Elle est fidelle à ce qu'elle aime ;

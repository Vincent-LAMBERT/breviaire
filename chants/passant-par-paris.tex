\header[h]{
    \headtitle{Passant par Paris} \label{passant-par-paris}
    %
    
    \insertComment{Chanson de marin amenée à Paris en 1870 par des canonniers.}{}
}

\enluminure{4}{\href{https://www.youtube.com/watch?v=DdmN2gqCHJ4}{P}}\\ 
$\left.\begin{tabular}{l}
\hspace{-0.4cm}
\textsc{assant} par Paris
\\
\hspace{-0.4cm}
Vidant la bouteille
\end{tabular}\right\rbrace$ bis
\\L'un de mes amis
\\Me dit à l'oreille
\\\\\hphantom{~~~~~~~~~~~~~~~~~~}Pom, pom, pom, pom, pom, pom, pom...
\\\textbf{Refrain :}\hphantom{~~~~} Le bon vin m'endort, l'amour me réveille,
\\\hphantom{~~~~~~~~~~~~~~~~~~}Le bon vin m'endort, l'amour me réveille encore.
\dualcol{
\bisdouble{L'un de mes amis}
{Me dit à l'oreille, ~~~~~~~}
Jean prends garde à toi
\\L'on courtise ta belle
\bisdouble{Jean prends garde à toi}
{L'on courtise ta belle}
Courtise qui voudra
\\Je me fie en elle
\bisdouble{Courtise qui voudra ~~~~}
{Je me fie en elle}
J'ai eu de son coeur
\\La fleur la plus belle
\bisdouble{J'ai eu de son coeur ~~~~}
{La fleur la plus belle}
Dans un grand lit blanc
\\Gréé de dentelles
\bisdouble{Dans un grand lit blanc}
{Gréé de dentelles}
J'ai eu trois garçons
\\Tous trois capitaines
\bisdouble{J'ai eu trois garçons}
{Tous trois capitaines ~}
L'un est à Bordeaux
\\L'autre à La Rochelle
\bisdouble{L'un est à Bordeaux}
{L'autre à La Rochelle}
L'plus jeune à Paris
\\Courtisant les belles
\bisdouble{Le plus jeune à Paris ~}
{Courtisant les belles}
Le père est ici
\\Tirant la ficelle
\bisdouble{Le père est ici}
{Tirant la ficelle ~~~~~~~~}
Quand il a trois sous
\\S'en va au bordel.
\bisdouble{Quand il a trois sous ~}
{S'en va au bordel.}
Quand il n'en a pas
\\S'en va boire bouteille.
}

\textbf{Pom, pom, pom, pom, pom, pom, pom...} 
\\\textbf{Le bon vin m'endort, l'amour me réveille,}
\\\textbf{Et quand vient l'aurore, l'amour me réveille encore !}
\breakpage
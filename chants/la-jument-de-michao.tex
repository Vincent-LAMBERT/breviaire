\header{
    \headtitle{La jument de Michao} \label{la-jument-de-michao}
    %
    \insertComment{Version bretonne par Kouerien (1973) d'une chanson bourguignonne "le loup le renard et le lièvre du XVe siècle.}{La version bourguignonne dérive de "Ai vist lo lop" en langue occitane datant du XIIe siècle.}
}

\enluminure{4}{\href{https://www.youtube.com/watch?v=iJPI1ohI_q8}{C}}\\
$\left.\begin{tabular}{l}
\hspace{-0.4cm}
\textsc{'est }dans dix ans je m'en irai
\\
\hspace{-0.4cm}
J'entends le loup et le renard chanter
\end{tabular}\right\rbrace$ bis\\
\\\\
\bisdoublespace{J'entends le loup, le renard, et la belette}
{J'entends le loup et le renard chanter}
C'est dans neuf ans je m'en irai
\\La jument de Michao a passé dans le pré
\bisdoublespace{La jument de Michao et son petit poulain}
{A passé dans le pré et mangé tout le foin}
\bisdoublespace{L'hiver viendra, les gars, l'hiver viendra}
{La jument de Michao, elle s'en repentira}
\textbf{On reprend du début en descendant les années une à une.}
\breakpage
\header{
    \headtitle{Dudule} \label{dudule}
    %
    
    \insertComment{Chanson du début du XXème siècle, chantée à Polytechnique Paris.}{}
}
\vspace{-0.3cm}
\enluminure{4}{\href{https://www.youtube.com/watch?v=6bf9fuI-q1g}{I}}{ls étaient} amoureux,
\\Ils s'aimaient tous les deux,
\\Ils étaient heureux.
\\Chaque soir, chaque matin,
\\Ils allaient au turbin
\\La main dans la main.
\\A l'atelier ses copines lui disaient:
\\Pourquoi tu l'aimes tant ton Dudule?
\\Il est pas beau, il est mal fait.
\\Mais elle, gentiment, répondait:
\\Z'en faites pas les amies,
\\Moi, c'que j'aime en lui:
\\\\\textbf{Refrain :}
\\C'est la grosse bite à Dudule
\\J'la prends, j'la suce, elle m'encule.
\\Ah! Mes amies, vous dire c'que c'est bon
\\Quand il m'la carre dans l'oignon!
\\C'est pas une bite ordinaire,
\\Quand il m'la fout dans l'derrière
\\Je m'sens soudain toute remplie
\\Du con jusqu'au nombril. Ah! Dudule!
\\\\Ca devait arriver:
\\Ils se sont mariés,
\\Ils ont convolé.
\\D'abord ça tournait rond,
\\Ils s'chatouillent le menton.
\\Mais lui file des gnons.
\\A l'atelier, ses copines lui disaient:
\\Pourquoi tu l'tues pas ton Dudule?
\\Il est pas beau, il t'a cirée.
\\Mais, elle, gentiment, répondait:
\\Z'en faites pas les amies,
\\Moi, c'que j'aime en lui:

\breakpage
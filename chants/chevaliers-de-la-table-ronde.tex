\header[v]{
    \section{Chevaliers de la table ronde} \label{chevaliers-de-la-table-ronde}
    %
    
    \insertComment{Premier enregistrement: Stello (1930). Les premières versions parlent d'une femme de chevalier alcoolique.}{}
    %Il s'agit de la même histoire de femme alcoolique qued'autres chansons comme fanchon par exemple
}

\noindent
\enluminure{3}{\href{https://www.youtube.com/watch?v=6RaZoWKM7ao}{C}}{hevaliers} de la Table Ronde
\\Goûtons voir si le vin est bon
\\Goûtons voir, oui oui oui
\\Goûtons voir, non non non
\\Goûtons voir si le vin est bon
\\\\S'il est bon s'il est agréable,
\\J'en boirai jusqu'à mon plaisir.
\\\\Et si le tonneau se débonde,
\\J'en boirai jusqu'à mon loisir.
\\\\Et s'il en reste quelques gouttes
\\Ce sera pour nous rafraîchir.
\\\\J'en boirai cinq à six bouteilles
\\Une femme sur les genoux.
%initialement il s'agit du mari
\\\\Pan, pan, pan, qui frappe à la porte?
\\Je crois bien que c'est mon amie.
\\\\Si c'est elle, que l'diable l'emporte
\\De venir troubler mon plaisir.
\\\\Si je meurs, je veux qu'on m'enterre
\\Dans une cave où y a du bon vin.
\\\\Les deux pieds contre la muraille
\\Et la tête sous le robinet.
\\\\Et les quatre plus grands ivrognes
\\Porteront les quatr' coins du drap.
\\\\Sur ma tombe, je veux qu'on inscrive:
\\"Ici git le roi des buveurs”.
\breakpage
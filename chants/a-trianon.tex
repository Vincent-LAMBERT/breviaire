\header{
    \headtitle{A trianon} \label{a-trianon}
    %
    \insertComment{Chanson composée dans les années 1920 par Fortugé, reprenant l'histoire d'un poème éponyme de Augusta Holmès en 1896 }{}
}

\enluminure{2}{\href{https://www.youtube.com/watch?v=A8xtQY3DjuI}{C}}{a s'passait} un jour à Trianon,
\\Dans la verdure et la bruyère.
\\Au milieu de ses petits moutons
\\Lucas embrassait sa bergère.
\\Pendant que la chatouillait le gars,
\\Lisette riait à tue-tête,
\\Et comme la censure
\\En ce temps là n’existait pas,
\\Dans les bosquets et les taillis,
\\On entendait ceci :
\\"Embrasse-moi le . . . Ho ! Ho !
\\Embrasse-moi le . . . Ha ! Ha !
\\Embrasse-moi le plus discrètement possible.
\\Je vais enfin toucher
\\Ton p’tit . . . Ho ! Ho !
\\Ton p’tit . . . Ha ! Ha !
\\Ton p’tit coeur sensible.
\\Ecartons les . . . Ho ! Ho !
\\Ecartons les . . . Ha ! Ha !
\\Ecartons les curieux de cet endroit paisible."
\\Et c’est ainsi que ça s’passait,
\\Tir’ la ridaine,
\\Tir’ la ridon,
\\Dans les jardins de Trianon.
\\\\La marquise en les voyant s’aimer,
\\Jalouse, vint troubler la fête.
\\Elle envoya Lison chez l’tripier
\\Chercher une chopine d’allumette.
\\L’enfant partit d’un pas guilleret.
\\Tous deux restèrent tête à tête.
\\Ce qui se passa
\\A ce moment-là,
\\On ne le sait pas.
\\Dans les bosquets et les taillis,
\\On entendait ceci :
\\"Je veux un gros . . . Ho ! Ho !
\\Je veux un gros . . . Ha ! Ha !
\\Je veux un gros bouquet, petit berger volage.
\\Je veux que tu me le mettes au . . . Ho ! Ho !
\\Je veux que tu me le mettes au . . . Ha ! Ha !
\\Me le mettes au corsage.
\\Je te tiens les . . . Ho ! Ho !
\\Je te tiens les . . . Ha ! Ha !
\\Je te tiens les mains pour t’jouer à être sage."
\\\\La fillette n’trouva quand elle revint
\\La marquise ni l’amant frivole.
\\Pour mourir, elle mit sur son pain
\\D’la saccharine et du pétrole.
\\Mais voici qu’à quelques temps de là,
\\Lucas revint à son idole.
\\Ce qui se passa
\\A ce moment là,
\\On ne le dit pas.
\\Dans les bosquets et les taillis,
\\On entendait ceci :
\\N’m’embrasse plus le . . . Ho ! Ho !
\\N’m’embrasse plus le . . . Ha ! Ha !
\\N’m’embrasse plus le soir au son du rossignol.
\\Car le marquis m’a donné sa . . . Ho ! Ho !
\\Car le marquis m’a donné sa . . . Ha ! Ha !
\\M’a donné sa parole,
\\De m’couper les . . . Ho ! Ho !

\breakpage
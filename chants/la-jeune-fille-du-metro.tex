\header{
    \headtitle{La jeune fille du métro} \label{la-jeune-fille-du-metro}
    %
    
    \insertComment{Chanson de Jean Rousselière (1933).}{}
}

\enluminure{4}{\href{https://www.youtube.com/watch?v=iNae44ZDknU}{C}}{'était une} jeune fille simple et bonne
\\Qui n'demandait rien à personne
\\Un soir dans l'métro y avait presse
\\Un jeune homme osa j'le confesse
\\Lui passer la main sur les...cheveux
\\Comme elle était gentille elle s'approcha un peu
\\\\Mais comme elle craignait pour sa robe
\\A ses attaques elle se dérobe
\\Sentant quelque chose qui la chatouille
\\De sa main elle tripatouille
\\Elle tombe sur une belle paire de...gants
\\Qu'le jeune homme à la main tenait négligemment
\\\\L'jeune homme vit l'mouv'ment d'la d'moiselle
\\Il s'rapprocha un peu plus d'elle
\\Et comme à chaque homme tout de suite
\\S'éveille le démon qui l'habite
\\Le jeune homme lui sortit sa...carte
\\Et lui dit "J'm'appelle Jules et j'habite rue Descartes"
\\\\L'métro continue son voyage
\\Elle se dit c'jeune homme n'est point sage
\\Elle sent quelque chose de pointu
\\Qui d'un air ferme et convaincu
\\Cherche à pénétrer dans son...coeur
\\Ah! Qu'il est doux d'aimer, quel frisson de bonheur.
\\\\Ainsi à Grenoble quand on s'aime %(ou Paris)
\\On peut se le dire sans problème
\\Peu importe le véhicule
\\N'ayons pas peur du ridicule
\\Dites simplement je t'en...prie
\\Viens donc à la maison manger des spaghettis.
\\
\breakpage
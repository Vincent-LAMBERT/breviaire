\header{
    \section{Le temps des cerises} \label{le-temps-des-cerises}
    %
    \insertComment{Paroles de Jean Baptiste Clément, musique de Antoine Renard.}{Chanson fortement associée à la commune de Paris.}
}

\enluminure{4}{\href{https://www.youtube.com/watch?v=ncs4WlWfIZo}{Q}}{uand} nous chanterons le temps des cerises
\\Et gai rossignol et merle moqueur
\\Seront tous en fête
\\Les belles auront la folie en tête
\\Et les amoureux du soleil au cœur
\\Quand nous chanterons le temps des cerises
\\Sifflera bien mieux le merle moqueur
\\\\Mais il est bien court le temps des cerises
\\Où l'on s'en va à deux cueillir en rêvant
\\Des pendants d'oreilles
\\Cerises d'amour aux robes pareilles
\\Tombant sous la feuille en gouttes de sang
\\Mais il est bien court le temps des cerises
\\Pendants de corail qu'on cueille en rêvant
\\\\Quand vous en serez au temps des cerises
\\Si vous avez peur des chagrins d'amour
\\Évitez les belles
\\Moi qui ne crains pas les peines cruelles
\\Je ne vivrai point sans souffrir un jour
\\Quand vous en serez au temps des cerises
\\Vous aurez aussi des peines d'amour
\\\\J'aimerai toujours le temps des cerises
\\C'est de ce temps-là que je garde au cœur
\\Une plaie ouverte
\\Et Dame Fortune, en m'étant offerte
\\Ne pourra jamais fermer ma douleur
\\J'aimerai toujours le temps des cerises
\\Et le souvenir que je garde au cœur
\breakpage
\header{
    \headtitle{Le chant des corsaires} \label{le-chant-des-corsaires}
    %
    
    \insertComment{Chant flamand du 17ème siècle.}{Jean Bart était un corsaire célèbre pour ses exploits au service de la France durant les guerres de Louis XIV.}
    %existe une version feminie ici : https://galilee.eedf.fr/wp/rassemblement-midipy/2019/07/28/le-chant-des-corsaires/
}
\vspace{-0.3cm}
\enluminure{4}{\href{https://www.youtube.com/watch?v=9zX9Y-rxSJA}{S}}\\ 
$\left.\begin{tabular}{l}
\hspace{-0.4cm}
\textsc{ont} des hommes de grand courage,
\\
\hspace{-0.4cm}
Ceux qui partiront avec nous
\end{tabular}\right\rbrace$ bis
\\Ils ne craindront point les coups,
\\Ni les naufrages, ni l'abordage,
\\Du péril seront jaloux
\\Tout ceux qui partiront avec nous. \bissimple
\\
\bisdouble{Ce seront de hardis pilotes,}
{Les gars que nous embarquerons.}
Fins gabiers et francs lurons
\\Je t'escamote, toute une flotte
\\Bras solides et coup d'oeil prompt
\\Tous les gars que nous embarquerons. \bissimple
\\
\bisdouble{Ils seront de fiers camarades,}
{Ceux qui navigueront à bord,}
Faisant feu babord, tribord,
\\Dans la tornade, des canonnades
\\Vainqueurs rentreront au port
\\Tout ceux qui navigueront à bord. \bissimple
\\
\bisdouble{Et les prises de tout tonnage}
{Nous ramènerons avec nous}
Et la gloire, et les gros sous,
\\Feront voyages, dans nos sillages,
\\Vent arrière, ou vent debout
\\Nous les ramènerons avec nous. \bissimple
\\
\bisdouble{Car c'est le plus vaillant corsaire}
{Qui donna l'ordre du départ.}
Vite en mer et sans retard.
\\Faisons la guerre à l'Angleterre,
\bisdouble{Car c'est le fameux Jean Bart,}
{Qui nous commandera le départ.}

\breakpage

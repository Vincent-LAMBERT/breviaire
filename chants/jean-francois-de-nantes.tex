\header{
    \section{Jean-François de Nantes} \label{jean-francois-de-nantes}
    %
    
    \insertComment{}{}
    %\insertComment{Chanson du début du 19ème siecle (probablement sous Napoléon).}{}
    % commenté pour cause de mise en page
}

\enluminure{4}{\href{https://www.youtube.com/watch?v=44FWZ03kWog}{C}}{'est Jean-François} de Nantes
\\Oué ! Oué ! Oué !
\\Gabier de la Fringante
\\Oh ! Mes boués ! Jean-Françoué !
%le mot Boué est valide
\\\\Débarqu'en fin d'campagne
\\Fier comme un roi d'Espagne
\\\\En vrac, dedans sa bourse...
\\Il a vingt mois de course...
\\\\Une montre, une chaîne...
\\Valant une baleine...
\\\\Branle-bas chez son hôtesse...
\\Bite et bosses et largesses...
\\\\La plus belle des servantes...
\\L'emmèn' dans sa soupente...
\\\\De conserve avec elle...
\\Navigue sur mer belle...
\\\\Et vidant la bouteille...
\\Tout son or appareille...
\\\\Montr' et chaîne s'envolent...
\\Mais il prend la vérole...
\\\\A l'hôpital de Nantes...
\\Jean-Françoué se lamente...
\\\\Et les draps de sa couche...
\\Déchire avec sa bouche...
\\\\Pauv' Jean-Françoué de Nantes...
\\Gabier de la Fringante...

\breakpage
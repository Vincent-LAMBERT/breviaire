\header{
    \headtitle{Alphonse du gros caillou} \label{alphonse-du-gros-caillou}
    %
    \insertComment{Un Alphonse est un homme entretenu par une femme.}{Texte initialment proclamé et publié en 1888 dans Monologues par Hyppolyte Lacombe. Le gros caillou était une maison close de l'époque prenant son nom d'un rocher détruit pour la construction des Invalides.}
}

\enluminure{4}{\href{https://www.youtube.com/watch?v=4sf898Unr2s}{J}}{'m'appell'Alphonse'}, j'n'ai pas d'nom de famille,
\\Parc'que mon pèr' n'en avait pas non plus,
\\Quant à ma mèr', c'était un'pauvre fille
\\Qui était née de parents inconnus.
\\On l'appelait Thérès', pas davantage,
\\Quoiqu'non mariés, c'étaient d'heureux époux ;
\\Et l'on disait quel beau petit ménage,
\\Que le ménage Alphons'du Gros Caillou !
\\\\Après trois ans, ils eur'nt enfin la chance,
\\Vu leur conduit', leurs bons antécédents,
\\D'pouvoir ouvrir un'maison d'tolérance
\\Et surtout cell'd'avoir eu quatre enfants.
\\Sur quatre enfants, Dieu leur donna trois filles
\\Qui ont servi dès qu'ell's ont pu chez nous ;
\\C'est que c'était une honnête famille,
\\Que la famille Alphons'du Gros Caillou !
\\\\Tout prospéra, mes soeurs aidant ma mère
\\Car elles eur'nt vite fait leur chemin ;
\\Moi-même aussi, et quelquefois mon père
\\S'il le fallait, nous y prêtions la main.
\\La clientèle était assez gentille,
\\Car elle avait grande confiance en nous ;
\\Ils s'en allaient disant ; quelle famille,
\\Que la famille Alphons'du Gros Caillou !
\breakpage
Moi j'travaillais dans la magistrature,
\\Le haut clergé, les gros officiants,
\\J'avais pour ça l'appui d'la préfecture
\\Où je comptais aussi quelques clients.
\\J'étais si beau qu'on m'prenait pour un'fille,
\\Tant j'étais tendre et caressant et doux
\\Aussi j'étais l'orgueil de la famille,
\\De la famille Alphons'du Gros Caillou !
\\\\Y avait des jours, fallait être solide,
\\Et le quinze août, fête de l'Empereur,
\\C'était chez nous tout rempli d'invalides,
\\De pontonniers, d'cuirassiers, d'artilleurs.
\\Car ce jour-là, le militair'godille
\\Et tous ces gens sortaient contents d'chez nous ;
\\Ils se disaient quelle belle famille,
\\Que la famille Alphons'du Gros Caillou !
\\\\Au-dehors nous comptions quelques pratiques
\\Ma mèr'servait les Dam's du Sacré Coeur,
\\Mes soeurs servaient Madam'de Metternich,
\\Mon pèr'servait la Maison de l'Emp'reur.
\\La clientèle était assez gentille,
\\Puis on avait grande confiance en nous
\\Et l'on disait : "Quelle sainte famille
\\Que la famille Alphons'du Gros Caillou"
\\\\Maint'nant ma mèr's'est r'tirée des affaires,
\\Moi j'continue mais c'est en amateur ;
\\Mes soeurs ont tout's épousé des notaires
\\Mon père est membr'de la Légion d'Honneur,
\\De notr'vertu la récompense brille
\\Et si notr'sort a pu fair'des jaloux,
\\On dit tout d'mêm'c'est un'belle famille,
\\Que la famille Alphons'du Gros Caillou !
\breakpage
\header{
    \headtitle{Les prisons de Nantes} \label{les-prisons-de-nantes}

    \insertComment{Inspirée de l'évasion du cardinal de Retz de 1654, la chanson est composée en Basse-Loire au 17ème siècle.}{} 
}

\enluminure{4}{\href{}{D}}{ans} les prisons de Nantes,
\\Lan di-gi-di-gidan, di-gidi, lan di, lan di-gi-di-gidan
\\Dans les prisons de Nantes,
\\Y'avait un prisonnier. ~~\bissimple
\\\\Personne ne le vint vouère,
\\Lan di-gi-di-gidan, di-gidi, lan di, lan di-gi-di-gidan
\\Personne ne le vint vouère,
\\Que la fille du geôlier ~~~~~~\bissimple
\dualcol{
Elle lui apporte à boire,
\\A boire et à manger. ~~~~~~~\bissimple
\\\\Et des chemises blanches,
\\Quand il veut en changer. \bissimple
\\\\Un jour il lui demande:
\\Mais que dit-on de moué ? \bissimple
\\\\On dit de vous en ville,
\\Que vous serez pendu. ~~~~~\bissimple
\\\\Mais s'il faut qu'on me pende
\\Déliez moi les pieds. ~~~~~~~~\bissimple
\\\\La fille était jeunette,
\\Les pieds lui a délié. ~~~~~~~~\bissimple
\\\\Le prisonnier alerte,
\\Dans la Loire s'est jeté. ~~\bissimple
\\\\A la première plonge
\\A manqué d'se noyer. ~~\bissimple
\\\\A la seconde plonge,
\\La Loire a traversé. ~~\bissimple
\\\\Des qu'il fut sur les rives,
\\Il se mit a chanter. ~~~~~~~~\bissimple
\\\\Je chante pour les belles,
\\Surtout celle du geôlier. ~~\bissimple
\\\\Si je reviens a Nantes,
\\Oui je l'épouserai ! ~~~~~~~~~\bissimple
\\\\Dans les prisons de Nantes,
\\Y'avait un prisonnier. ~~~~~\bissimple
\\}

\breakpage
\header{
    \section{Les Canuts} \label{les-canuts}
    %
    
    \insertComment{Célèbre chant de lutte, au même titre que Le temps des cerises ou Bella ciao, ce chant traditionnel de Lyon a été écrit en 1894 par Aristide Bruant.}{La révolte des canuts désigne plusieurs soulèvements ouvriers ayant lieu à Lyon, en France, en 1831 puis 1834 et 1848. Il s'agit de l'une des grandes insurrections sociales du début de l’ère de la grande industrie.}
}
%
\enluminure{3}{\href{https://www.youtube.com/watch?v=TRpz3Yu_Vvo}{P}}~\\
$\left.\begin{tabular}{l}
\hspace {-0.4cm}
\textsc{our} chanter Veni Creator
\\
\hspace{-0.4cm}
Il faut une chasuble d'or
\end{tabular}\right\rbrace$ bis
\\Nous en tissons pour vous, grands de l'église
\\Et nous pauvres canuts, n'avons pas de chemise
\\C'est nous les canuts
\\Nous sommes tout nus !
\\\bisdouble{Pour gouverner, il faut avoir}
{Manteaux ou rubans en sautoir.}
\\Nous en tissons pour vous grands de la terre
\\Et nous, pauvres canuts, sans drap on nous enterre
\\C'est nous les canuts
\\Nous sommes tout nus !
\\\bisdouble{Mais notre règne arrivera}
{Quand votre règne finira :}
\\Nous tisserons le linceul du vieux monde,
\\Car on entend déjà la révolte qui gronde
\bisdouble{C'est nous les canuts}
{Nous n'irons plus nus !}

\breakpage
\header{
    \section{Lo Dauphinois} \label{lou-dauphinois}
    %
    \insertComment{"Magnaud" désigne les dauphinois de souche et "miron" se rapporte aux turripinnois, habitants de la Tour du Pin.}{}%{Dicton: "Susquié que vou peto pi yo qu'son cu se fa on golé à le côtés", littéralement: "Celui qui veut péter plus haut que son cul se fait un trou aux côtes".}
}
%https://fr.scribd.com/document/89647849/Lou-dauphinois-sont-de-Magnauds-Terribles

\enluminure{4}{\href{}{B}}{rave} magno de vodrè ben vo djire
\\Quôque coplé que vo ne savons pa
\\Mé mon crayon pora té vo l'écrire
\\Mé mon gojye pora té lo thantor
\\De voua d'abour comenchye ma complainta
\\En vo parlant du miron de la Tôr
\\Car m'en vityé que ne pourton pas plainta
\\Quand y zen bin tjui aruza lu fôr
\\\\\textbf{Refrain :}
\\Lo Dauphinois sont de magnauds terribles
\\Qu'on doble ner et que fo to tremblo
\\Mé si sont fours n'en s'ont pas moins resibles
\\U amon tjui bien bère et s'amuso
\\\\Pré de la Tôr, San Didyi la cassoula
\\Y a de magno ké fo pa plézanta
\\Et ni aré su toble que na fioula
\\Don co de poing nen foutron vingt pe bas
\\Ne poussa pas la nya de cela sourta
\\Lo co de poing ne kuto pa grand liôr
\\Et vo aro vite prindre la pourta
\\Si pe malhu, vo la ayan piata
\\\\E fo léchye de Zallya lo renaille
\\E fo léchye éto lo brégogniôr
\\Sé no fallye comenchye lo rekoya
\\De vo zassure é vo tiendro trop târ
\\E vodro mya parlo de cele filles
\\Qué ne font pa lo bonhur du papa
\\Mé qu'omon mya, e derri le tharmille
\\Vo zambrachye, magno de Montcarra
\breakpage
\\\\Lo Dauphinois an de bien bèle rôtes
\\Hivèr, été, te pou le féquento
\\Mais si on zôe te passare tu tôtes
\\A Nivolas passe sans t'arreta
\\Mé si te vou te deguirye ta vesta
\\Mé si te vou te fore dessampillye
\\Si ta on zôe na véprena de resta
\\Vo la passa vé la cayon de Ruy
\\\\Nyon n'a zamé mizye notre alagne
\\Car é aiyan mé don fameaux gayôr
\\Qu'en fé tremblo é plane zi montagne
\\Et vo zétyo passo su le muraille
\\De Dolamya, San Didye, la Thapella
\\Filles et garçons n'en po tro fra u ziâ
\\Mé lo pi fâo de cela ribambelle
\\Son, den sé seur, lo zânes de Céchya
\\\\\\\textbf{Traduction :}
\\Braves magnauds, je voudrais bien vous dire
\\Quelque couplet que point vous ne savez
\\Mais mon crayon pourra-t-il vous l'écrire
\\Mais mon gosier pourra-t-il le chanter ?
\\Je vais d'abord commencer ma complainte
\\En vous parlant des mirons de la Tour
\\Car en voilà qui ne portent pas plainte
\\Quand ils ont tous bien arrosé leur four.
\\\\\textbf{Refrain :}
\\Les Dauphinois sont des magnauds terribles
\\Aux nerfs d'acier et qui font tout trembler
\\Mais s'ils sont forts, sont joyeux au possible
\\Ils aiment tous bien boire et s'amuser
\breakpage
\\\\Près de la Tour, Saint Didier la casserole
\\Y a des magnauds qu'y faut pas plaisanter
\\Et s'il n'y avait sur la table qu'une fiole
\\D'un coup de poing, en mettraient 20 par terre
\\Ne poussez pas votre compagnie ainsi
\\Les coups de poing ne coûtent pas très cher
\\Et vous auriez en hâte pris la porte
\\Si par malheur, les aviez écrasés
\\\\Il faut laisser de Jallieu les grenouilles
\\Il faut laisser aussi les Bergusiens
\\Et s'il fallait commencer le recueil
\\Je vous assure que ce serait trop long
\\Ne vaut il pas mieux parler de ces filles
\\Qui ne font pas le bonheur du papa
\\Mais qui aiment mieux derrière les charmilles
\\Vous embrasser, magnaud de Montcarra
\\\\Les Dauphinois ont de bien belles routes
\\Hiver, été, tu peux les fréquenter
\\Mais si un jour tu peux passer sur toutes
\\A Nivolas passe sans t'arrêter
\\Mais si tu veux te déchirer la veste
\\Mais si tu veux qu'on arrache tes habits
\\Et si un jour, l'après-midi te reste
\\Va la passer chez les cochons de Ruy
\\\\Personne n'a jamais mangé nos noisettes
\\Car il y avait plus d'un fameux gaillard
\\Qui font trembler plaines et montagnes
\\Et vous seriez passés dessus les murs
\\De Dolomieu, Saint Didier, La Chapelle
\\Filles et garcons n'ont pas trop froid aux yeux
\\Mais les plus fous de cette ribambelle
\\Sont, j'en suis sûr, les ânes de Cessieu
\breakpage
\header{
    \section{La Coupo Santo } \label{la-coupo-santo}
    %
    \insertComment{La Coupo santo, c'est-à-dire la Coupe sainte, est une coupe en argent que les félibres catalans offrirent aux félibres provençaux lors d’un banquet qui se tint à Avignon le 30 juillet 1867, en remerciement de l’accueil réservé au poète catalan Victor Balaguer, exilé politique en Provence. Cette coupe est l’œuvre du sculpteur et statuaire Louis Guillaume Fulconis et de l’argentier Jarry.}{La chanson de la coupe fut écrite pour commémorer cet événement par Frédéric Mistral sur la musique d’un chant de Noël oeuvre du frère Sérapion. Elle est devenue depuis l'hymne de la Provence et même l'un des hymnes de l'Occitanie.}
}

\enluminure{4}{\href{https://www.youtube.com/watch?v=YdGTqaNV88s}{P}}{rouvençau} , veici la Coupo
\\Que nous vèn di Catalan
\\A-de-rèng beguen en troupo
\\Lou vin pur de noste plan
\\\\\textbf{Refrain :}
\\Coupo Santo
\\E versanto
\\Vuejo à plen bord,
\\Vuejo abord
\\Lis estrambord
\\E l'enavans di fort !
\\\\D'un vièi pople fièr e libre
\\Sian bessai la finicioun ;
\\E, se toumbon li felibre,
\\Toumbara nosto nacioun
% Je commente les couplets en +, en général il n'y a que le premier, deuxième et dernier qui sont chantés (Lucie)
%\\\\D'uno raço que regreio
%\\Sian bessai li proumié gréu ;
%\\Sian bessai de la patrìo
%\\Li cepoun emai li priéu.
%\\\\Vuejo-nous lis esperanço
%\\E li raive dóu jouvènt,
%\\Dóu passat la remembranço,
%\\E la fe dins l'an que vèn,
%\\\\Vuejo-nous la couneissènço
%\\Dóu Verai emai dóu Bèu,
%\\E lis àuti jouïssènço
%\\Que se trufon dóu toumbèu
%\\\\Vuejo-nous la Pouësìo
%\\Pèr canta tout ço que viéu,
%\\Car es elo l'ambrousìo,
%\\Que tremudo l'ome en diéu
\\\\Pèr la glòri dóu terraire
\\Vautre enfin que sias counsènt.
\\Catalan, de liuen, o fraire,
\\Coumunien tóutis ensèn !
\\
\breakpage
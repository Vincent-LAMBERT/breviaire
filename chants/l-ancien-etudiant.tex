\header{
    \section{L'ancien étudiant} \label{l-ancien-etudiant}
    %
    \insertComment{}{}
}

\enluminure{4}{\href{https://xavier.hubaut.info/paillardes/texte2.htm\#ANCI}{P}}{uisons} amis, l'oubli dans nos ivresses,
\\Que bière et vin soient pour nous bienvenus,
\\L'alcool nous pousse aux lascives caresses,
\\Sine Baccho, dit-on, friget Vénus.
\\Mes chers amis, aux heures de marasme,
\\Soyez-en sûrs, ce vin fortifi-ant,
\bisdoublespace{Vous remplira d'ardeur et d'enthousiasme}
{Voilà l'avis d'un vieil étudi-ant. }
\\\\Pour s'abstenir de fumer et de boire,
\\Les tempérants ne s'en portent pas mieux,
\\Suivons toujours les conseils de l'histoire,
\\Soyons au moins dignes de nos aïeux.
\\Le vieil Horace a chanté le Falerne,
\\Le bon Bergson le vin fortifi-ant !
\bisdoublespace{Tout mon plaisir, c'est la taverne}
{Voilà l'avis d'un vieil étudi-ant. }
\\\\Surtout fuyons, fuyons comme la peste
\\Ces péroreurs appelés tempérants,
\\Pour eux le vin, le bon gîte et le reste
\\Sont des plaisirs presque déshonorants
\\De ces cons-là méprisons les disciples
\\Pâles crevés aux regards larmoyants 
\bisdoublespace{Et flanquons-nous des tamponnes multiples}
{Voilà l'avis d'un vieil étudi-ant.}
\\\\S'épouvanter de quelque vague buse,
\\N'est que le fait d'un indigne froussard
\\Et je prétends qu'il faut que l'on s'amuse
\\Pour s'éviter bien des regrets plus tard.
\\O mes amis ! Si par quelque magie,
\\Je reprenais ma jeunesse à l'instant,
\bisdouble{Je resuivrais le chemin de l'orgie, }
{Voilà l'avis d'un vieil étudi-ant. }

\breakpage
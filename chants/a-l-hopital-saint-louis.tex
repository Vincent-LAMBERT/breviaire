\header{
    \section{A l'hôpital Saint-Louis} \label{a-l-hopital-saint-louis}
    %
    
    \insertComment{Deux chansons voisines dans l'Anthologie hospitalière et latinesque (1911) : Le cheveu dans la merde et Le délicat.}{En 1607, Henri IV, fonde la Maison de la Santé, suite à une épidémie de peste. Renomée l'hôpital Saint-Louis en mémoire de Louis IX, mort de la peste en 1270.}
    
    % L'hôpital, y compris les combles et les annexes fut terminé en 1612. Mais les salles ne furent ouvertes aux malades qu'en 1616, et c'est deux ans plus tard qu'une épidémie (de peste ?) se déclara.
}

\enluminure{4}{\href{https://www.youtube.com/watch?v=QAp4lg4nS54}{A}}{l'hôpital} Saint-Louis
\\Dans la fosse aux humeurs
\\C'est là que je me réjouis
\\A m'faire des tartines de beurre
\\\\\textbf{Refrain :}
\\Moi j'm'en fous ! J'bouffe de tout!
\\Si j'mange bien, si j'chie peu
\\C'est afin que rien n'se perde
\\Si j'suis dégouté d'la merde
\\C'est qu'j'y ai trouvé un ch'veu
\\Deux cheveux...
\\\\Sur les bords de la Seine
\\J'rencontre un chien crevé
\\Je lui tire les vers du nez
\\Et j'les bouffe à l'italienne
\\\\Mon frère est poitrinaire
\\Et dégueule toute la nuit
\\Si je couche à côté d'lui
\\C'est pour mieux gober ses glaires
\\\\Tous les mois c'est l'usage
\\Ma femme saigne du con
\\Si je suce ses tampons
\\Ça épargn'le le blanchissage
\breakpage
\dualcol{
\\\\Quand mon gosse a la chiasse
\\J'lui lèche le trou du cul
\\Et comme je suis barbu
\\J'en attrape plein les moustaches
\\\\Quand je vois mon vieil oncle
\\J'l'embrasse la bouche en coeur
\\C'est pour mieux sucer l'humeur
\\Qui coule des ses furoncles
\\\\Quand un vieil invalide
\\A fait cinq ou six lieues
\\Je lui lèche le tour des yeux
\\Et je suc' ses chancres putrides
\\\\Le pus syphillitique
\\L'urine des chaud'pisseux
\\Sont des breuvages délicieux
\\Et des nectars angéliques
\\\\Ce que les femmes enceintes
\\Rejettent en accouchant
\\Est un met appétissant
\\Que j'garde pour la s'maine sainte
\\\\Ce que dans les pissotières
\\Un type a dégueulé
\\Je m'empresse de le bouffer
\\Avec une petite cuillère
\\\\Quand l'facteur du village
\\A fini sa journée
\\Je lui lèche la plante des pieds
\\Ça remplace le fromage
\\\\Quand un vésicatoire
\\Suppure et rend du jus
\\Moi, je pose ma langue dessus
\\J'pense ainsi manger et boire
\\\\Messieurs, si ma ballade
\\Vous donne le hoquet
\\Dégueulez dans le baquet
\\J'aime aussi la dégueulade
}

\breakpage
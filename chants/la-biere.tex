\header{
    \headtitle{La bière} \label{la-biere}
    %
    
    \insertComment{Chanson d'Antoine Clesse (1866).}{}
}
%
\enluminure{4}{\href{https://www.youtube.com/watch?v=paAt3bdoNQg}{E}}{lle} a vraiment d'une bière flamande
\\L'air avenant, l'éclat et la douceur.
\\Joyeux Wallons, elle nous affriande
\\Et le Faro trouve en elle une soeur.
\\\\\textbf{Refrain :}
\\A plein verre, mes bons amis,
\\En la buvant, il faut chanter la bière,
\\A plein verre, mes bons amis,
\\Il faut chanter la bière du pays.
\\\\Voyez là-bas la kermesse en délire :
\\Les pots sont pleins, jouez ménestriers !
\\Quels jeux bruyants et quels éclats de rire !
\\Ce sont encore des "Flamands" des Teniers !
\\\\Aux souverains, portant tout haut leur plaintes,
\\Bourgeois jaloux, des droits de la cité,
\\Nos francs aïeux, tout en vidant leur pinte,
\\Fondaient les arts avec la liberté.
\\\\Quand leurs tribuns, à l'attitude altière,
\\Faisaient sonner le tocsin des beffrois,
\\Tous ces fumeurs, tous ces buveurs de bière,
\\Savaient combattre et mourir pour leurs droits.
\\\\Belges, chantons à ce refrain à boire !
\\Peintres, guerriers qui nous illustrent tous,
\\Géants couchés dans leur linceul de gloire,
\\Vont s'éveiller, pour redire avec nous.
\\\\Salut à toi, bière limpide et blonde !
\\Je tiens mon verre, et le bonheur en main,
\\Ah ! J'en voudrais verser à tout le monde,
\\Pour le bonheur de tout le genre humain.

\breakpage
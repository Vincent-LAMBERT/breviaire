\header[v]{
    \section{Les moines de Saint Bernardin} \label{les-moines-de-saint-bernardin}
    %
    
    \insertComment{Texte de la deuxième moitié du 14ème siècle par Eusache Deschamps.}{Saint Bernardin est la fête des moines lubriques et se tient le 20 mai.}
}
%

\enluminure{4}{\href{https://www.youtube.com/watch?v=CSrQLHZ6M_c}{N}}{ous} sommes les moines de Saint-Bernardin, \bissimple
\\Nous nous couchons tard et nous levons matin, \bissimple
\\Pour aller à matines vider quelques flacons,
\\Voilà qui est bon, est bon, est bon.
\\\\\textbf{Refrain :}
\bisdoublespace{Et voilà la vie, la vie, la vie, la vie chérie. Ah! Ah!}
{Et voilà la vie que tous les moines font.}
\\\\Pour notre déjeuner du bon chocolat \bissimple
\\Et du bon café que l'on nomme moka \bissimple
\\Et la tarte sucrée et les marrons de Lyon
\\Voilà c' qu'est bon, et bon et bon!
\\\\Pour notre dîner, de bons petits oiseaux, \bissimple
\\Que l'on nomme cailles, bécasses, ou perdreaux, \bissimple
\\De la fine andouillette et la tranche de jambon
\\Voilà qui est bon, est bon, est bon.
\\\\Pour notre coucher, dans un lit aux draps blancs, \bissimple
\\Une jeune nonne, de prês de vingt ans, \bissimple
\\Qui a la taille bien faite, et les nichons bien ronds,
\\Voilà qui est bon, est bon, est bon.
\\\\La nuit, tous ensemble, nous nous enculons, \bissimple
\\Jusqu'au jour, ensemble, nous buvons, buvons,\bissimple
\\Après, sous la table, nous roulons et dormons,
\\Voilà qui est bon, est bon, est bon.
\\\\Si c'est là la vie que tous les moines font, \bissimple
\\Je me ferai moine, avec ma Jeanneton,\bissimple
\\Le soir, dans ma chambrette, je lui chatouillerai le bouton,
\\Voilà qui est bon, est bon, est bon.

\breakpage
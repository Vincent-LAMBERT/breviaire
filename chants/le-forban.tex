\header{
    \section{Le Forban} \label{le-forban}
    %
    
    \insertComment{Daterait de la deuxième partie du 18ème siècle, écrite par des bagnards de Brest.}{}
    %Semble être la chanson traditionelle. ? Oui mais encore x) Pas rajouté parce que ça veut rien dire pour moi
}

\enluminure{4}{\href{https://www.youtube.com/watch?v=vOyg086Lw2s}{A}}{ moi} forban que m'importe la gloire
\\Les lois du monde et qu'importe la mort ?
\\Sur l'océan j'ai planté ma victoire
\\Et bois mon vin dans une coupe d'or.
\\Vivre d'orgies est ma seule espérance
\\Le seul bonheur que j'aie pu conquérir
\\Si sur les flots j'ai passé mon enfance
\\C'est sur les flots qu'un forban doit mourir.
\\\\\textbf{Refrain :}
\\Vin qui pétille, femme gentille
\\Sous tes baisers brûlants d'amour, oui d'amour
\\Plaisir bataille vive la canaille
\\Je bois, je chante et je tue tour à tour.
\\\\Peut-être au mât d'une barque étrangère
\\Mon corps un jour servira d'étendard.
\\Et tout mon sang rougira la galère
\\Aujourd'hui fête, et demain le bazar.
\\Allons, esclave, allons debout mon brave
\\Buvons le vin et la vie à grand pot
\\Aujourd'hui fête, et puis demain peut-être
\\Ma tête ira s'engloutir dans les flots.
\\\\Peut-être un jour, par un coup de fortune,
\\Je saisirai l'or d'un beau galion
\\Riche à pouvoir vous acheter la lune,
\\Je m'en irai vers d'autres horizons.
\\Là, respecté tout comme un gentilhomme,
\\Moi qui ne suis qu'un forban, qu'un bandit,
\\Je pourrai, comme le fils d'un roi, tout comme,
\\Mourir, peut-être, dedans un bon lit.


\breakpage
\header{
    \section{Chanson à virer} \label{chanson-a-virer}
    %
    \insertComment{Traduction de Drunken sailor proposée par Henry Jacques.}{Une chanson à virer permettait jusque dans les années 1920 de garder le rythme pour remonter l'ancre à bord des bateaux. Cette opération pouvait demander plusieurs heures d'effort à au moins deux hommes.}
}

\enluminure{4}{\href{https://www.youtube.com/watch?v=gir1kgziRdc&list=UUEsNVIAs_A8AqmebJ-5GAtA}{H}}{ardi} les gars l'ancre est dans les fonds
\\Hardi les gars maillon par maillon
\\Hardi les gars nous l'arracherons les gars si nous virons.
\\\\\textbf{Refrain :}
\\Encor'et hop et vire
\\Encor'et hop et vire
\\Encor'et hop et vire
\\Vire encore un coup.
\\\\C'est pas l'moment les gars d'être saouls
\\C'est pas l'moment d'avoir les bras mous
\\C'est pas l'moment d'plier les g'noux les gars faut virer tout.
\\\\L'ancre est à pic on va déraper
\\L'ancre est à pic la mer a lâché
\\L'ancre est à pic des mains, des pieds les gars il faut virer.
\\\\Encore un coup c'est pour le retour
\\Encore un coup enlèv’le plus lourd
\\Encore un coup c'est l'dernier tour les gars virons toujours.


\breakpage
\header{
    \section{Alexandre} \label{alexandre}
    %
    \insertComment{Publiée en 1627 dans le Parnasse des muses.}{}
}

\enluminure{4}{\href{https://xavier.hubaut.info/paillardes/ancien.htm#ALEX}{A}}{lexandre} dont le nom
\\A rempli la terre,
\\N'aimait pas tant le canon
\\Qu'il faisait le verre
\\Si le grand Mars des guerriers
\\S'est acquis tant de lauriers
\\Que devons, vons,vons,
\\Que pouvons, vons, vons,
\\Que devons nous faire
\\Sinon de bien boère?
\dualcol{
\\\\Quand la mer rouge apparût
\\Aux yeux de Grégoire,
\\Aussitôt ce buveur crut
\\Qu'il n'avait qu'à boire;
\\Moïse fut bien plus fin
\\Voyant que ce n'était vin
\\Il la pa, il la pa,
\\Il la sa, sa, sa,
\\Il la passa toute
\\Sans en boire goutte.
\\\\Le bonhomme Gédéon
\\Faisait des merveilles,
\\Aussi n'usait, se dit-on
\\Rien que de bouteilles.
\\Servons nous donc aujourd'hui
\\De bouteilles comme lui
\\Et faisons, sons, sons,
\\Et faisons, sons, sons,
\\Et faisons la guerre
\\À grands coups de verres.
\\\\Loth qui fut homme de bien.
\\Se plaisait à boère,
\\Dieu ne lui en disait rien
\\Il le laissait faire
\\Et puis quand il était saoul
\\Il s'endormait comme nous
\\Dans un' ca, ca, ca,
\\Dans un vern', vern', vern'
\\Dans une caverne
\\Près de la taverne.
\\\\Noé pendant qu'il vivait,
\\Patriarche digne,
\\Savait bien comme on buvait
\\Du fruit de la vigne;
\\De peur qu'il ne bût de l'eau
\\Dieu lui fit faire un bateau
\\Pour chercher, cher, cher,
\\Pour trouver, ver, ver,
\\Pour chercher refuge
\\Au temps du déluge.
}
\breakpage
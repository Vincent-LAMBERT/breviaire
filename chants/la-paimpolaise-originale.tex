\header{
    \section{La paimpolaise (Originale)} \label{la-paimpolaise-originale}
    %
    \insertComment{Ecrite en 1784 à Paris par un dinannais inconnu. Reprise l'an suivant par Félix Mayol}{}
}

\enluminure{4}{\href{https://www.youtube.com/watch?v=YwUQ9F-iYdY}{Q}}{uittant} ses genêts et sa lande
\\Quand le Breton se fait marin
\\En allant aux pêches d'Islande
\\Voici quel est le doux refrain
\\Que le pauvre gars
\\Fredonne tout bas
\\\\J'aime Paimpol et sa falaise
\\Son église et son Grand Pardon
\\J'aime surtout la Paimpolaise
\\Qui m'attend au pays breton
\\\\Le brave Islandais, sans murmure
\\Jette la ligne et le harpon
\\Puis, dans un relent de saumure
\\Il se couche dans l'entrepont
\\Et le pauvre gars
\\Fredonne tout bas
\\\\Je serais bien mieux à mon aise
\\Devant mon joli feu d'ajonc
\\À côté de la Paimpolaise
\\Qui m'attend au pays breton
\\\\Mais souvent l'océan qu'il dompte
\\Se réveillant lâche et cruel
\\Et lorsque que le soir on se compte
\\Bien des noms manquent à l'appel
\\Et le pauvre gars
\\Soupire tout bas
\\\\Pour servir la flotte française
\\Puisqu'il faut plus d'un moussaillon
\\J'en causerai à ma Paimpolaise
\\Qui m'attend au pays breton
\breakpage